% !TEX encoding = IsoLatin2

% LaTeX Template Version 3.1,  JuLY 2011
% by Dr. Andreas Drauschke (andreas.drauschke@technikum-wien.at) and Dr. Susanne Teschl (susanne.teschl@technikum-wien.at)
% minor adaptations by Harald Stockinger (harald.stockinger@technikum-wien.at)
%
% Adaptions for MR done by M. Widrich (BMR6) in January 2012, tested by W. Kubinger in March 2012
%
% Fragen zu den BMR- und MMR-Vorlagen bitte an Wilfried Kubinger (kubinger@technikum-wien.at) richten.
%
% Angepasst an MMR4, SS2013: 27.2.2013, WK
%


\documentclass[a4paper,bibtotoc,oneside]{scrbook}
% For short papers the documentclass "scrartcl" is sufficient. In this case the highest sectioning structure is "section" ("chapter" does not exist).
% \documentclass[a4paper,bibtotoc,oneside]{scrartcl}

\usepackage{hyperref}

\usepackage[ansinew]{inputenc}
\usepackage[T1]{fontenc}
\usepackage[ngerman,english]{babel}
\selectlanguage{english}
\usepackage{amsmath,amssymb,amsfonts,amstext}

\usepackage{fancyhdr}
\lfoot[\fancyplain{}{}]{\fancyplain{}{}}
\rfoot[\fancyplain{}{}]{\fancyplain{}{}}
\cfoot[\fancyplain{}{\footnotesize\thepage}]{\fancyplain{}{\footnotesize\thepage}}
\lhead[\fancyplain{}{\footnotesize\nouppercase\leftmark}]{\fancyplain{}{}}
\chead{}
\rhead[\fancyplain{}{}]{\fancyplain{}{\footnotesize\nouppercase\sc\leftmark}}

\usepackage{color}

\usepackage{helvet}
\renewcommand{\familydefault}{cmss}

\usepackage[pdftex]{graphicx}

\usepackage{harvard}

\usepackage{array}

\setlength{\textheight}{225mm}
\setlength{\textwidth}{1.05\textwidth}


\tolerance = 9999
\sloppy

% Anpassung einiger "Uberschriften
\renewcommand\figurename{Figure}
\renewcommand\tablename{Table}

% Abbildungen, Gleichungen und Tabellen werden fortlaufend nummeriert
\renewcommand\thefigure{\arabic{figure}}
\renewcommand\thetable{\arabic{table}}
\renewcommand\theequation{\arabic{equation}}
\usepackage{remreset}
\makeatletter
  \@removefromreset{figure}{chapter}
  \@removefromreset{table}{chapter}
  \@removefromreset{equation}{chapter}
\makeatother

%Zum korrekten Formatieren von Verzeichnissen
\usepackage{tocloft}
\renewcommand{\cftfigpresnum}{Figure~}
\renewcommand{\cfttabpresnum}{Table~}
\renewcommand{\cftfigaftersnum}{:}
\renewcommand{\cfttabaftersnum}{:}
\setlength{\cftfignumwidth}{2.5cm}
\setlength{\cfttabnumwidth}{2.5cm}
\setlength{\cftfigindent}{0cm}
\setlength{\cfttabindent}{0cm}

\begin{document}

%Festlegen des Zitier-Standards
\bibliographystyle{HarvardFHTWMR_V1_2e}%Zitierstandard FH Technikum Wien, Studiengang Mechatronik/Robotik
\citationstyle{dcu}%Correct citation-style (Harvardand, ";" between citations, "," between author and year)
\citationmode{abbr}%use "et al." with first citation
    \newcommand{\citepic}[1]{(Source: \protect\cite{#1})}%Zitat: Bild
    \newcommand{\citefig}[2]{(Source: \protect\cite{#1}, p. #2)}%Zitat: Bild aus Dokument
    \newcommand{\citefigm}[2]{(Source: taken with modification from \protect\cite{#1}, p. #2)}%Zitat: modifiziertes Bild aus Dokument
    \newcommand{\citep}{\citeasnoun}%In-Line Zitiat entweder mit \citep{} oder \citeasnoun{}
    \newcommand{\acessedthrough}{Available at:}%F�r URL-Angabe
    \newcommand{\acessedthroughp}{Available through:}%F�r URL-Angabe (Gesch�tzte Datenbank, Zugriff durch FH)
    \newcommand{\acessedat}{Accessed}%F�r URL-Datum-Angabe
    \newcommand{\singlepage}{p.}%F�r Seitenangabe (einzelne Seite)
    \newcommand{\multiplepages}{pp.}%F�r Seitenangabe (mehrere Seiten)
    \newcommand{\chapternr}{Ch.}%F�r Kapitelangabe
    \renewcommand{\harvardand}{\&}%Harvardand in Zitaten
    \newcommand{\abstractonly}{Abstract only}
    \newcommand{\edition}{~edition}%Edition -> note, that you have to write "edition = {2nd},"!

\pagestyle{fancy}

% title page:
\thispagestyle{empty}
\begin{picture}(0,0)
\color{white}\sffamily
\put(-101,-749){\includegraphics[width=1.002\paperwidth, height=\paperheight]{BM_2011.pdf}}
\put(220,-670){\includegraphics[width=0.5\textwidth]{FHTW_Logo_4c.pdf}}
\put(-30, -20){\bfseries\huge BACHELOR'S PAPER}
% insert degree program:
\put(-30,-50){\Large Degree Program XXX}


\put(-32,-150){
\begin{minipage}{14cm}
\bfseries\huge
% insert title:
Title Title Title Title Title Title Title Title Title Title
\end{minipage}
}
% insert author:
\put(-30,-250){\large By: Degree First name Surname, Degree}
% insert student ID:
\put(-30,-270){\large Student Number: XXXXXXXXXX}
% insert supervisor:
\put(-30,-310){\large Supervisor: Degree First name Surname, Degree}
\put(-30,-350){\large Vienna, \today}
\color{black}
\end{picture}


\newpage

\section*{Declaration}\thispagestyle{empty}
I confirm that this paper is entirely my own work. All sources and quotations have been fully acknowledged in the appropriate places
with adequate footnotes and citations. Quotations have been properly acknowledged and marked with appropriate punctuation.
The works consulted are listed in the bibliography. This paper has not been submitted to another examination panel in the same or a
similar form, and has not been published. I declare that the present paper is identical to the version uploaded.
\\[5\baselineskip]
\rule{5cm}{0.2pt}\hfill\rule{5cm}{0.2pt}\\
\phantom{Datum }Place, Date\hfill Signature\hspace{15mm}
\newpage





% German abstract:
\section*{Kurzfassung}\thispagestyle{empty}
Text Text Text Text Text Text Text Text Text Text Text Text Text Text Text Text Text Text Text Text Text Text Text Text ...
\\ \vfill
% Please insert 3-5 German keywords that characterize the thesis:
\paragraph*{Schlagw{\"o}rter:} Schlagwort 1, Schlagwort 2, Schlagwort 3, Schlagwort 4, Schlagwort 5


\newpage

% English abstract:
\section*{Abstract}\thispagestyle{empty}
Text Text Text Text Text Text Text Text Text Text Text Text Text Text Text Text Text Text Text Text Text Text Text Text ...
\\ \vfill
% Please insert 3-5 English keywords that characterize the thesis:
\paragraph*{Keywords:} Keyword 1, Keyword 2, Keyword 3, Keyword 4, Keyword 5
\newpage

\section*{Acknowledgements}
\thispagestyle{empty}
Text Text Text Text Text Text Text Text Text Text Text Text Text Text Text Text Text Text Text Text Text Text Text Text ...
\newpage

\tableofcontents\thispagestyle{empty}
\newpage

\setcounter{page}{1}
\chapter[First Chapter]{Heading of the first chapter}

Text Text Text Text Text Text Text Text Text Text Text Text Text Text Text Text Text Text Text Text Text Text Text Text ...

\section[First Section]{Heading of the first section}

Text Text Text Text Text Text Text Text Text Text Text Text Text Text Text Text Text Text Text Text Text Text Text Text ...

\subsection[First Subsection]{Heading of the first subsection}

Text Text Text Text Text Text Text Text Text Text Text Text Text Text Text Text Text Text Text Text Text Text Text Text ...

\subsubsection[First Sub-subsection]{One more level}

Text Text Text Text Text Text Text Text Text Text Text Text Text Text Text Text Text Text Text Text Text Text Text Text ...

\selectlanguage{ngerman}%Erm�glichung der einfacheren Schreibweise von deutschen Umlauten und Anf�hrungszeichen
\chapter{Anmerkungen zum MR-Zitierstandard}

\section{Verwendung von BibTeX}

Das Literaturverzeichnis wird automatisch generiert. Die Quellenangaben befinden sich in der Datei "`*.bib"'. Zur einfachen Recherche der Literaturquellen-Daten empfiehlt sich auch \textit{google-scholar} (\url{scholar.google.de}). Dazu wird unter den "`Einstellungen"' (rechts oben) die Option  "`\textbf{Bibliographiemanager}: Links zum Importieren von Literaturverweisen in BibTeX anzeigen."' aktiviert. Danach kann eine Literaturquelle gesucht und durch einen Klick auf "`In BibTeX importieren"' die Informationen per Copy\& Paste "ubernommen werden.\\

Kompiliert wird "uber bibtex mit pdflatex oder PS bzw. DVI.
F"ur diese Vorlage z.B. mit pdflatex$\rightarrow$bibtex$\rightarrow$bibtex$\rightarrow$pdflatex$\rightarrow$pdflatex. Die URL-Darstellung und die PDF Eigenschaften sowie die r"omische Nummerierung im PDF vor dem Hauptteil werden mit dem hypersetup bzw. \textbackslash pagenumbering\{\} generiert.\\


\section{Zitate "`im Text"'}

In-Line Zitate k"onnen mit \textbackslash citeasnoun\{Quelle\} oder \textbackslash citep\{Quelle\} durchgef"uhrt werden. Bei mehreren Autoren sind die Quellen-Namen nach der Jahreszahl zu ordnen.

Beispiele f"ur Zitate "`im Text"': nach \citeasnoun{Braun07} und \citeasnoun{Kastner11} oder \citep{Kessler11}.\\

\section{Zitate "`nicht direkt im Text"'}

"`Normale"' Zitate in Klammer werden mit \textbackslash cite\{Quelle\} erzeugt. Bei mehreren Autoren sind die Quellen-Namen durch einen Beistrich zu trennen und nach der Jahreszahl zu ordnen.

Beispiele f"ur Zitate "`nicht direkt im Text"': \cite{Technikum11} bzw. \cite{Bach82,Zettler98,Astrom01}.

\section{Mehrere Werke des selben Autors im selben Jahr}

Bei mehreren Zitaten vom selben Autor im selben Jahr ist die Jahreszahl mit "`a"', "`b"', etc. zu erweitern (siehe *.bib-Datei).

Beispiele f"ur mehrere Zitate des selben Autors im selben Jahr: \cite{Aangerman09a} und \cite{Aangerman09b}.\\

\section{Abk"urzung mit "`et al."'}

Bei mehr als zwei Autoren wird automatisch mit "`et al."' abgek"urzt: \cite{Zettler98}.

\section{Bildbeschriftungen}

Es gibt folgende Markos f"ur Quellenangabe bei fremden Fotos/Bildern:

\begin{itemize}
\item Referenz auf Foto/Bild: \textbackslash citepic\{Hemetsberger07\} $\overset{wird zu}{\Longrightarrow}$ \citepic{Hemetsberger07}
\item Referenz auf Foto/Bild aus Dokument: \textbackslash citefig\{Lund92\}\{99\} $\overset{wird zu}{\Longrightarrow}$ \citefig{Lund92}{99}
\item Referenz auf modifiziertes Foto/Bild aus Dokument: \textbackslash citefigm\{Lund92\}\{150\} $\overset{wird zu}{\Longrightarrow}$ \citefigm{Lund92}{150}
\end{itemize}\ \\

\section{Verf"ugbare Medientypen}
Folgende Medientypen sind zur Zeit m"oglich:
\begin{itemize}
\item Bachelorarbeit/Projektbericht/etc.: \cite{Baldinger10,Piringer11}.
\item Buch: \cite{Aangerman09a,Aangerman09b}.
\item Datenblatt, Leitfaden: \cite{Anglia10,Atmel11}.
\item E-Abstract (Nur Abstract verf"ugbar): \cite{Astrom01}.
\item E-Book: \cite{Kastner11}.
\item E-Book (Zugang durch FH): \cite{Kessler11}.
\item E-Magazin oder E-Journal: \cite{Lund92,Zinner07}.
\item E-Magazin oder E-Journal (Zugang durch FH): \cite{Bach82}.
\item Edited Book: \cite{Braun07,Braun10}.
\item Kapitel in einem Edited Book: \cite{Samson70,Smith75}.
\item Masterthese/Dissertation: \cite{Pohn10,Humenberger11}.
\item Normen: \cite{ISO98}.
\item Paper, Konferenzbeitrag, Journalartikel: \cite{Zettler98,Gesztesy00}.
\item Patent: \cite{Anderson10}.
\item Photographien, Bilder: \cite{Hemetsberger07}.
\item Photographien, Bilder (online): \cite{Dean08}.
\item Website: \cite{Technikum11}.
\item Zeitung: \cite{Slapper05}.
\item Zeitung (online): \cite{Chittenden03}.
\end{itemize}

\selectlanguage{english}%Ende des deutschen Teils

% Literaturverzeichnis
\bibliography{Vorlage_BT_BMR_WS2013_Literatur}
\newpage

% Abbildungsverzeichnis und Tabellenverzeichnis
\begingroup
    \renewcommand*{\addvspace}[1]{}
    \phantomsection
    \addcontentsline{toc}{chapter}{\listfigurename}
    \listoffigures
    \newpage
    \phantomsection
    \addcontentsline{toc}{chapter}{\listtablename}
    \listoftables
\endgroup


% If you use the document class "scrartcl" you need to use  \addsec{List of Abbreviations} instead of \addchap{List of Abbreviations}
\addchap{List of Abbreviations}
\hspace{-17mm}\begin{tabular}{>{\raggedleft}p{0.2\linewidth} p{0.75\linewidth} p{0.1\linewidth}}
www & World Wide Web \\
URL & Uniform Resource Locator
\end{tabular}

\begin{appendix}
\chapter[First Appendix]{Heading of the first appendix}

Text Text Text Text Text Text Text Text Text Text Text Text Text Text Text Text Text Text Text Text Text Text Text Text ...


\chapter[Second Appendix]{Heading of the second appendix}

Text Text Text Text Text Text Text Text Text Text Text Text Text Text Text Text Text Text Text Text Text Text Text Text ...

\end{appendix}

\end{document}
